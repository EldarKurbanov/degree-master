\chapter{Анализ}\label{ch:ch2}

\section{Среда выполнения}\label{sec:ch2/sec1}

Проанализировав требования к среде выполнения, были представлены следующие варианты реализации:
\begin{enumerate}[beginpenalty=10000] % https://tex.stackexchange.com/a/476052/104425
  \item Собственная реализация среды на основе микросервисной клиент-серверной архитектуры;
  \item Использование стороннего фреймворка Robot Operating System (ROS).
\end{enumerate}

\subsection{Собственная реализация}
Данный вариант включает в себя реализацию следующего: 
\begin{enumerate}[beginpenalty=10000] % https://tex.stackexchange.com/a/476052/104425
  \item Сервер - хранилище сообщений от микросервисов. В его обязанности будут входить: приём, передача, согласование формата сообщений, удаление устаревших данных.
  \item Сами микросервисы. Помимо их основной работы, они должны принимать с сервера входные данные и отправлять выходные данные для последующей работы остальных сервисов.
\end{enumerate}

Схема представленного варианта изображена на Рисунке~\cref{fig:microservice}.

\begin{figure}[ht]
    \centerfloat{
        \includegraphics[scale=0.7]{microservice}
    }
    \caption{Пример архитектуры собственной реализации среды выполнения для робота на основе микросервисной архитектуры}\label{fig:microservice}
\end{figure}

Данный вариант был отвергнут в следствии существования более быстрого и удобного варианта реализации, описанного в следующей секции.

\subsection{Фреймворк Robot Operating System}
Robot Operating System или сокращённо ROS предоставляет удобные и мощные функции, помогающие разработчикам в таких задачах, как передача сообщений различного типа, распределение вычислений между компьютерами, повторное использование кода и реализация современных алгоритмов для роботов~\cite[с.~7]{ros}. В общем случае, ROS представляет собой инструмент, позволяющий связывать несколько независимых программных модулей при
помощи сервисов и узлов, которые могут передавать друг другу сообщения в различном формате. Структура ROS представлена на Рисунке~\cref{fig:ros-struct}~\cite[с.~19]{ros}.

\begin{figure}[ht]
    \centerfloat{
        \includegraphics[scale=0.9]{ros-struct}
    }
    \caption{Общая структура Robot Operating System}\label{fig:ros-struct}
\end{figure}

Большими преимуществами использования данного фреймворка является возможность передачи сообщений по локальной сети и обширная библиотека уже реализованного ПО, которое можно без относительно больших затрат по времени интегрировать в свой собственный проект. На момент написания данной работы глобальный репозиторий ROS Index насчитывает 2455 подключенных к нему сторонних репозиториев и 6927 пакетов. Диаграмму соответствия пакетов в репозитории с версиями ROS\footnote{Версия, используемая в данной работе - ROS Melodic} можно увидеть на Рисунке~\cref{fig:ros-repo}~\cite{ros-repo}.

\begin{figure}[ht]
    \centerfloat{
        \includegraphics[scale=0.7]{ros-repo}
    }
    \caption{Диаграмма соответствия глобального репозитория ROS Index и версий ROS}\label{fig:ros-repo}
\end{figure}

\subsubsection{Концепции ROS}
Ниже приведён список концепций рассматриваемого фреймворка:

\begin{itemize}[beginpenalty=10000] % https://tex.stackexchange.com/a/476052/104425
  \item \textbf{Узел} --- это процесс, выполняющий вычисления. Каждый узел написание с использованием клиентских библиотек ROS. Используя методы связи, узлы могут общаться друг с другом заранее определённым форматом сообщений и обмениваться данными. Для этого создаются узлы-подписчики, и узлы-публикаторы;
  \item \textbf{Мастер} --- обеспечивает регистрацию и работоспособность запущенных узлов;
  \item \textbf{Сообщение} --- простая структура данных, содержащая типизированное поле, которое может содержать целый набор данных, отправляемых на другой узел. Помимо стандартных типов сообщений\footnote{Такие как целые числа, с плавающей точкой, логический тип, строковый...} возможна отправка заранее обозначенных собственных типов сообщений;
  \item \textbf{Топик} --- именованная шина данных, используемая узлами для отправки сообщений. Публикующий и подписанный узел не знают о существовании друга друга. Благодаря тому, что каждый топик имеет уникальное имя, любой узел может получить доступ к данному топику и отправляет через него данные, при условии соблюдении заранее оговорённых передаваемых типов, данным топиком;
  \item \textbf{Сервисы} --- реализация удалённого вызова процедур\footnote{RPC} в ROS. В некоторых случаях модель связи публикации и подписки может не
подходить. В этих случаях и применяют взаимодействия в виде сервисов (схема запрос/ответ), при котором один узел может запросить выполнение процедуры для другого узла, ожидая какого-то обязательного ответа\footnote{В случае использования схемы с подписчиками и публикаторами доставка сообщений и ответ
не гарантируются}~\cite[с.~20]{ros}.
\end{itemize}

\section{Обеспечение необходимых данных}
\subsection{Информация об окружающем пространстве}
Для предоставления информации об окружающем пространстве в режиме реального времени со всех сторон робота было принято решение об установке прибора лазерного сканирования, реализующего технологию LiDAR. Такое устройство представляет собой дальномер оптического диапазона, который замеряет угол и расстояние до точки, получая таким образом полярные координаты.

Существует два основных типа LiDAR'ов:

\begin{enumerate}[beginpenalty=10000] % https://tex.stackexchange.com/a/476052/104425
  \item 3D LiDAR;
  \item 2D LiDAR.
\end{enumerate}

Первый позволяет получить 3D картинку. Обычно такой LiDAR оснащён подвижным лазером, который довольно долго сканирует перед собой окружающую местность. Однако, уже сейчас можно найти 3D LiDAR, которые сканируют с довольно быстрой скоростью~\cite[с. 308]{lidar-3d}. Примером результата такой работы может стать картинка, изображённая на Рисунке~\cref{fig:lidar-3d}. На сегодняшний день такие LiDAR'ы являются довольно дорогими устройствами.

\begin{figure}[ht]
    \centerfloat{
        \includegraphics[scale=1.0]{lidar-3d}
    }
    \caption{Пример визуализации данных 3D LiDAR'а, представленном на выставке CEATEC 2017 компанией Panasonic Японии}\label{fig:lidar-3d}
\end{figure}

Второй, соответственно, создаёт двухмерное облако точек, которое также можно визуализировать в виде картинки, пример которой изображён на Рисунке~\cref{fig:lidar-2d}. Такой LiDAR обычно сканирует область вокруг себя и имеет угол обзора 360 градусов. Лазер также является подвижным, но только в этот раз он просто движется вокруг своей оси~\cite[с. 610]{lidar-2d}.

\begin{figure}[ht]
    \centerfloat{
        \includegraphics[scale=1.0]{lidar-2d}
    }
    \caption{Пример визуализации данных 2D LiDAR'а YDLiDAR X4}\label{fig:lidar-2d}
\end{figure}

\subsection{Альтернатива LiDAR}
В качестве альтернативы, предложенной в предыдущем пункте, можно рассмотреть иной способ получения информации об окружающем пространстве - использование камеры глубины. Одними из самых известных представителей являются камеры Xbox 360 Kinect, а также Intel RealSense Depth Camera D455, изображённые на Рисунке~\cref{fig:lidar-alts}~\cite{realsense,kinect-picture}.

\begin{figure}[ht]
    \centerfloat{
        \hfill
        \subcaptionbox[Intel® RealSense™ Depth Camera D455]{Внешний вид Intel® RealSense™ Depth Camera D455\label{fig:realsense}}{%
            \includegraphics[width=0.5\linewidth]{realsense}}
        \hfill
        \subcaptionbox[Xbox 360 Kinect]{Внешний вид Xbox 360 Kinect\label{fig:kinect}}{%
            \includegraphics[width=0.5\linewidth]{kinect}}
        \hfill
    }
    \legend{}
    \caption[Камеры глубины]{Одни из самых известных камер глубины на рынке}\label{fig:lidar-alts}
\end{figure}

Камеры глубины в отличии от привычных видеокамер могут соотносить пиксели не только с цветом, но и с расстоянием до объекта в этой точке, как это показано на Рисунке~\cref{fig:depth-example}~\cite{depth-cameras}. К сожалению, данный способ получения информации об окружающем пространстве не позволит получать полную картину вокруг робота в следствии маленького угла обзора представленных камер сравнительно с лазерным сканером из предыдущего пункта данной работы.

\begin{figure}[ht]
    \centerfloat{
        \includegraphics[scale=0.5]{depth-example}
    }
    \caption{Пример изображения с камеры глубины Xbox 360 Kinect.}\label{fig:depth-example}
\end{figure}

\subsection{Изображение с видеокамеры}
На компьютере NVIDIA Jetson Xavier NX, на котором будет работать нейросеть, существует два варианта подключения видеокамеры, которые будут рассмотрены в пунктах ниже.

\subsubsection{USB подключение}

Такой вариант подключения является самым популярным и без проблемным, так как на рынке имеется большое количество универсальных видеокамер с подключением по USB для любых целей. Для задач данной работы отлично подойдёт видеокамера Logitech C720 с разрешением видеосъёмки 1280x720 пикселей, изображённая на Рисунке~\cref{fig:usb-camera}~\cite{usb-camera}.

\begin{figure}[ht]
    \centerfloat{
        \includegraphics[scale=0.27]{usb-camera}
    }
    \caption{Внешний вид USB видеокамеры Logitech C720}\label{fig:usb-camera}
\end{figure}

\subsubsection{CSI подключение}

Данный способ подключения обладает как преимуществами, так и недостатками в сравнении с классическим USB подключением. Сравнительный анализ можно найти в Таблице~\cref{fig:usb-camera}~\cite{compare-csi-usb}.

\begin{table} [htbp]
    \centering
    \begin{threeparttable}% выравнивание подписи по границам таблицы
        \caption{Сравнение способов подключения видеокамеры к компьютеру NVIDIA Jetson Xavier NX}\label{tab:CameraCompare}%
        \begin{tabular}{| p{4cm} || p{6cm} | p{6cm}l |}
            \hline
            Рассматриваемый аспект       & \centering USB 3.0 подключение                                                 & \centering CSI подключение                                                     & \\
            \hline
            Доступность                              & \centering  Широкий выбор модельного ряда                             & \centering Единственная совместимая камера                     & \\
            \hline
            Ширина канала данных           & \centering  400 МБ/с                                                                       & \centering  320 МБ/с по одной линии (всего 4 линии)           & \\
            \hline
            Надёжность                              & \centering  USB порт, который относительно тяжело сломать & \centering Легко переламывающийся шлейф и коннектор & \\
            \hline
            Максимальная длина кабеля & \centering  Менее 5 метров                                                            & \centering  Менее 30 сантиметров                                          & \\
            \hline
            <<Горячее>> подключение    & \centering  Поддерживается                                                          & \centering  Не поддерживается                                              & \\
            \hline
            Габариты устройства              & \centering  Как правило, небольшие: 73x32x66 мм                      & \centering  Крошечный размер: 25x24x9 мм                          & \\
            \hline
        \end{tabular}
    \end{threeparttable}
\end{table}

Компьютер NVIDIA Jetson Xavier NX обладает совместимостью с единственной CSI камерой модели Sony IMX219, изображённая на Рисунке~\cref{fig:usb-camera}. Она оснащена 8 мегапиксельной матрицей с фокусным расстоянием 33 мм, а также имеет режим видеосъёмки в разрешении 1920x1080 пикселей с частотой кадров 30\footnote{1080p 30fps}~\cite{csi-camera}.

\begin{figure}[ht]
    \centerfloat{
        \includegraphics[scale=1.3]{csi-camera}
    }
    \caption{Внешний вид CSI видеокамеры Sony IMX219}\label{fig:csi-camera}
\end{figure}

По итогу сравнения, приоритет был дан на CSI подключение для ускорения обработки изображения, так как ресурсы мобильного компьютера сильно ограничены.

\section{Шасси робота}
Шасси робота определяет его <<мобильность>> и способность преодолевать различные препятствия. При выборе шасси необходимо искать компромиссы. С одной стороны, оно не должно быть большим, чтобы была возможность проезда в узких местах и не должно быть маленьким чтобы уместить всё оборудование на безопасном расстоянии друг от друга. 

В основной своей массе, по своей подвижной части шасси подразделяются на гусеничные и колёсные. Для робота было выбрано гусеничное шасси TS100, изображённое на Рисунке~\cref{fig:chassis}~\cite{chassis}. Такой выбор сделан чтобы отработать движения на гусеницах, которые более устойчивы к неровной поверхности дороги\footnote{В будущем предполагается использование робота в полевых условиях}.

\begin{figure}[ht]
    \centerfloat{
        \includegraphics[scale=0.4]{chassis}
    }
    \caption{Внешний вид выбранного гусеничного шасси для робота TS100}\label{fig:chassis}
\end{figure}

Данное шасси имеет следующие характеристики~\cite{chassis}:
\begin{enumerate}[beginpenalty=10000] % https://tex.stackexchange.com/a/476052/104425
  \item Материал: алюминиевый сплав;
  \item Размер: 275x190x95 мм;
  \item Вес: 1100г.
\end{enumerate}

Шасси дополнительно укомплектовано двумя экземплярами электродвигателя DT25-370, которое изображено на Рисунке~\cref{fig:motor}~\cite{motor}.

\begin{figure}[ht]
    \centerfloat{
        \includegraphics[scale=0.27]{motor}
    }
    \caption{Внешний вид электродвигателя DT25-370}\label{fig:motor}
\end{figure}

Двигатель имеет следующие характеристики~\cite{motor}:
\begin{enumerate}[beginpenalty=10000] % https://tex.stackexchange.com/a/476052/104425
  \item Максимальная скорость: 150 оборотов в минуту;
  \item Допустимая нагрузка: 3 кг;
  \item Рабочее напряжение: 9 вольт;
  \item Датчики Холла: 2 шт;
  \item Максимальное потребление тока: 4.5 А.
\end{enumerate}

\section{Рассмотрение аналогов}
К известным аналогам разрабатываемого робота, созданных на базе такой же платформы Jetson можно причислить роботов от самой компании NVIDIA: JetBot и Kaya. Оба эти робота были созданы для демонстрации возможностей данного одноплатного компьютера.

\subsection{NVIDIA Kaya}
Данная модель компактного мобильного автономного робота была представлена на технологической конференции GTC 2019 и в первую очередь предназначается для работы с программным обеспечением Isaac SDK. Робот представлен на Рисунке~\cref{fig:kaya}.

\begin{figure}[ht]
    \centerfloat{
        \includegraphics[scale=1.0]{kaya}
    }
    \caption{Внешний вид робота NVIDIA Kaya}\label{fig:kaya}
\end{figure}

Аппаратно данный робот помимо самого Jetson NANO включает в себя пластиковый корпус на трёх колёсах (печатаемый на 3D принтере), 3D камеру LiDAR Intel Real Sense и систему управления. Общая стоимость аппаратной части на момент написания данной ВКР\footnote{июнь 2022} составляет 812.87\$~\cite{kaya}.

На компьютер Jetson NANO помимо ОС Ubuntu 18.04 LTS устанавливается ПО Isaac SDK и Isaac SIM. Isaac SDK - это открытая платформа NVIDIA для интеллектуальных роботов. Она предоставляет большой набор мощных алгоритмов, базирующихся на GPU вычислениях\footnote{вычисления на видеокарте} для навигации и управления.

На данном роботе можно запускать различные готовые примеры такие как ручное управление с игрового контроллера Playstation 4, автономное следование
за AprilTag, распознавание объектов на нейронной сети DetectNetv2 и алгоритм SLAM (основан на GMapping)~\cite{isaac-kaya}.

\subsection{NVIDIA JetBot}

JetBot был представлен на той же конференции, что NVIDIA Kaya и является гораздо более доступным вариантом (цена 422.99\$ на момент написания работы) для создания DIY робота (также он в отличии от Kaya имеется в розничной продаже одним комплектом и его не нужно собирать по частям из разных магазинов). NVIDIA JetBot изображён на Рисунке~\cref{fig:jetbot}~\cite{jetbot}.

\begin{figure}[ht]
    \centerfloat{
        \includegraphics[scale=1.0]{jetbot}
    }
    \caption{Внешний вид робота NVIDIA JetBot}\label{fig:jetbot}
\end{figure}

Аппаратно он состоит из всё той же Nvidia Jetson Nano, двух электромоторов с драйвером в комплекте и CSI видеокамеры Sony IMX219. 

Программная часть поставляется готовым образом на базе Ubuntu 18.04 в формате ISO для прошивки MicroSD карты.

Из доступных примеров имеется простое ручное управление через кнопки на экране с возможностью прямой трансляции изображения видеокамеры на экран в браузере и нейросеть, автономное движение по поверхности с распознанием препятствий и пропастей в окружающем пространстве при помощи нейронной сети на основе получаемого видеосигнала, также имеется функция следования робота за определённым целевым объектом~\cite{jetbot-examples}.

Для выравнивания изображения по-центру используется команда \verb+\centerfloat+, которая является во
многом улучшенной версией встроенной команды \verb+\centering+.

\section{Длинное название параграфа, в котором мы узнаём как сделать две картинки с~общим номером и названием}\label{sec:ch2/sect2}

А это две картинки под общим номером и названием:
\begin{figure}[ht]
    \begin{minipage}[b][][b]{0.49\linewidth}\centering
        \includegraphics[width=0.5\linewidth]{knuth1} \\ а)
    \end{minipage}
    \hfill
    \begin{minipage}[b][][b]{0.49\linewidth}\centering
        \includegraphics[width=0.5\linewidth]{knuth2} \\ б)
    \end{minipage}
    \caption{Очень длинная подпись к изображению,
        на котором представлены две фотографии Дональда Кнута}
    \label{fig:knuth}
\end{figure}

Те~же~две картинки под~общим номером и~названием,
но с автоматизированной нумерацией подрисунков:
\begin{figure}[ht]
    \centerfloat{
        \hfill
        \subcaptionbox[List-of-Figures entry]{Первый подрисунок\label{fig:knuth_2-1}}{%
            \includegraphics[width=0.25\linewidth]{knuth1}}
        \hfill
        \subcaptionbox{\label{fig:knuth_2-2}}{%
            \includegraphics[width=0.25\linewidth]{knuth2}}
        \hfill
        \subcaptionbox{Третий подрисунок, подпись к которому
            не~помещается на~одной строке}{%
            \includegraphics[width=0.3\linewidth]{example-image-c}}
        \hfill
    }
    \legend{Подрисуночный текст, описывающий обозначения, например. Согласно
        ГОСТ 2.105, пункт 4.3.1, располагается перед наименованием рисунка.}
    \caption[Этот текст попадает в названия рисунков в списке рисунков]{Очень
        длинная подпись к второму изображению, на~котором представлены две
        фотографии Дональда Кнута}\label{fig:knuth_2}
\end{figure}

На рисунке~\cref{fig:knuth_2-1} показан Дональд Кнут без головного убора.
На рисунке~\cref{fig:knuth_2}\subcaptionref*{fig:knuth_2-2}
показан Дональд Кнут в головном уборе.

Возможно вставлять векторные картинки, рассчитываемые \LaTeX\ <<на~лету>>
с~их~предварительной компиляцией. Надписи в таких рисунках будут выполнены
тем же~шрифтом, который указан для документа в целом.
На~рисунке~\cref{fig:tikz_example} на~странице~\pageref{fig:tikz_example}
представлен пример схемы, рассчитываемой пакетом \verb|tikz| <<на~лету>>.
Для ускорения компиляции, подобные рисунки могут быть <<кешированы>>, что
определяется настройками в~\verb|common/setup.tex|.
Причём имя предкомпилированного
файла и~папка расположения таких файлов могут быть отдельно заданы,
что удобно, если не~для подготовки диссертации,
то~для подготовки научных публикаций.
\begin{figure}[ht]
    \centerfloat{
        \ifdefmacro{\tikzsetnextfilename}{\tikzsetnextfilename{tikz_example_compiled}}{}% присваиваемое предкомпилированному pdf имя файла (не обязательно)
        \input{Dissertation/images/tikz_scheme.tikz}

    }
    \legend{}
    \caption[Пример \texttt{tikz} схемы]{Пример рисунка, рассчитываемого
        \texttt{tikz}, который может быть предкомпилирован}\label{fig:tikz_example}
\end{figure}

Множество программ имеют либо встроенную возможность экспортировать векторную
графику кодом \verb|tikz|, либо соответствующий пакет расширения.
Например, в GeoGebra есть встроенный экспорт,
для Inkscape есть пакет svg2tikz,
для Python есть пакет tikzplotlib,
для R есть пакет tikzdevice.

\section{Пример вёрстки списков}\label{sec:ch2/sec3}

\noindent Нумерованный список:
\begin{enumerate}
    \item Первый пункт.
    \item Второй пункт.
    \item Третий пункт.
\end{enumerate}

\noindent Маркированный список:
\begin{itemize}
    \item Первый пункт.
    \item Второй пункт.
    \item Третий пункт.
\end{itemize}

\noindent Вложенные списки:
\begin{itemize}
    \item Имеется маркированный список.
          \begin{enumerate}
              \item В нём лежит нумерованный список,
              \item в котором
                    \begin{itemize}
                        \item лежит ещё один маркированный список.
                    \end{itemize}
          \end{enumerate}
\end{itemize}

\noindent Нумерованные вложенные списки:
\begin{enumerate}
    \item Первый пункт.
    \item Второй пункт.
    \item Вообще, по ГОСТ 2.105 первый уровень нумерации
          (при необходимости ссылки в тексте документа на одно из перечислений)
          идёт буквами русского или латинского алфавитов,
          а второй "--- цифрами со~скобками.
          Здесь отходим от ГОСТ.
          \begin{enumerate}
              \item в нём лежит нумерованный список,
              \item в котором
                    \begin{enumerate}
                        \item ещё один нумерованный список,
                        \item третий уровень нумерации не нормирован ГОСТ 2.105;
                        \item обращаем внимание на строчность букв,
                        \item в этом списке
                              \begin{itemize}
                                  \item лежит ещё один маркированный список.
                              \end{itemize}
                    \end{enumerate}

          \end{enumerate}

    \item Четвёртый пункт.
\end{enumerate}

\section{Традиции русского набора}

Много полезных советов приведено в материале
<<\href{https://kostyrka.ru/main/ru/typesetting-and-typography-crash-course-by-kostyrka/}{Краткий курс благородного набора}>>
(автор А.\:В.~Костырка).
Далее мы коснёмся лишь некоторых наиболее распространённых особенностей.

\subsection{Пробелы}

В~русском наборе принято:
\begin{itemize}
    \item единицы измерения, знак процента отделять пробелами от~числа:
          10~кВт, 15~\% (согласно ГОСТ 8.417, раздел 8);
    \item \(\tg 20\text{\textdegree}\), но: 20~{\textdegree}C
          (согласно ГОСТ 8.417, раздел 8);
    \item знак номера, параграфа отделять от~числа: №~5, \S~8;
    \item стандартные сокращения: т.\:е., и~т.\:д., и~т.\:п.;
    \item неразрывные пробелы в~предложениях.
\end{itemize}

\subsection{Математические знаки и символы}

Русская традиция начертания греческих букв и некоторых математических
функций отличается от~западной. Это исправляется серией
\verb|\renewcommand|.
\begin{itemize}
    %Все \original... команды заранее, ради этого примера, определены в Dissertation\userstyles.tex
    \item[До:] \( \originalepsilon \originalge \originalphi\),
          \(\originalphi \originalleq \originalepsilon\),
          \(\originalkappa \in \originalemptyset\),
          \(\originaltan\),
          \(\originalcot\),
          \(\originalcsc\).
    \item[После:] \( \epsilon \ge \phi\),
          \(\phi \leq \epsilon\),
          \(\kappa \in \emptyset\),
          \(\tan\),
          \(\cot\),
          \(\csc\).
\end{itemize}

Кроме того, принято набирать греческие буквы вертикальными, что
решается подключением пакета \verb|upgreek| (см. закомментированный
блок в~\verb|userpackages.tex|) и~аналогичным переопределением в
преамбуле (см.~закомментированный блок в~\verb|userstyles.tex|). В
этом шаблоне такие переопределения уже включены.

Знаки математических операций принято переносить. Пример переноса
в~формуле~\eqref{eq:equation3}.

\subsection{Кавычки}
В английском языке приняты одинарные и двойные кавычки в~виде ‘...’ и~“...”.
В~России приняты французские («...») и~немецкие („...“) кавычки (они называются
«ёлочки» и~«лапки», соответственно). ,,Лапки`` обычно используются внутри
<<ёлочек>>, например, <<... наш гордый ,,Варяг``...>>.

Французкие левые и правые кавычки набираются
как лигатуры \verb|<<| и~\verb|>>|, а~немецкие левые
и правые кавычки набираются как лигатуры \verb|,,| и~\verb|‘‘| (\verb|``|).

Вместо лигатур или команд с~активным символом "\ можно использовать команды
\verb|\glqq| и \verb|\grqq| для набора немецких кавычек и команды \verb|\flqq|
и~\verb|\frqq| для набора французских кавычек. Они определены в пакете
\verb|babel|.

\subsection{Тире}
%  babel+pdflatex по умолчанию, в polyglossia надо включать опцией (и перекомпилировать с удалением временных файлов)
Команда \verb|"---| используется для печати тире в тексте. Оно может быть
несколько короче английского длинного тире (подробности в~документации
русификации babel). Кроме того, команда задаёт небольшую жёсткую отбивку
от~слова, стоящего перед тире. При этом, само тире не~отрывается от~слова.
После тире следует такая же отбивка от текста, как и~перед тире. При наборе
текста между словом и командой, за которым она следует, должен стоять пробел.

В составных словах, таких, как <<Закон Менделеева"--~Клапейрона>>, для печати
тире надо использовать команду \verb|"--~|. Она ставит более короткое,
по~сравнению с~английским, тире и позволяет делать переносы во втором слове.
При~наборе текста команда \verb|"--~| не отделяется пробелом от слова,
за~которым она следует (\verb|Менделеева"--~|). Следующее за командой слово
может быть  отделено от~неё пробелом или перенесено на другую строку.

Если прямая речь начинается с~абзаца, то перед началом её печатается тире
командой \verb|"--*|. Она печатает русское тире и жёсткую отбивку нужной
величины перед текстом.

\subsection{Дефисы и переносы слов}
%  babel+pdflatex по умолчанию, в polyglossia надо включать опцией (и перекомпилировать с удалением временных файлов)
Для печати дефиса в~составных словах введены две команды. Команда~\verb|"~|
печатает дефис и~запрещает делать переносы в~самих словах, а~команда \verb|"=|
печатает дефис, оставляя \TeX ’у право делать переносы в~самих словах.

В отличие от команды \verb|\-|, команда \verb|"-| задаёт место в~слове, где
можно делать перенос, не~запрещая переносы и~в~других местах слова.

Команда \verb|""| задаёт место в~слове, где можно делать перенос, причём дефис
при~переносе в~этом месте не~ставится.

Команда \verb|",| вставляет небольшой пробел после инициалов с~правом переноса
в~фамилии.

\section{Текст из панграмм и формул}

Любя, съешь щипцы, "--- вздохнёт мэр, "--- кайф жгуч. Шеф взъярён тчк щипцы
с~эхом гудбай Жюль. Эй, жлоб! Где туз? Прячь юных съёмщиц в~шкаф. Экс-граф?
Плюш изъят. Бьём чуждый цен хвощ! Эх, чужак! Общий съём цен шляп (юфть) "---
вдрызг! Любя, съешь щипцы, "--- вздохнёт мэр, "--- кайф жгуч. Шеф взъярён тчк
щипцы с~эхом гудбай Жюль. Эй, жлоб! Где туз? Прячь юных съёмщиц в~шкаф.
Экс-граф? Плюш изъят. Бьём чуждый цен хвощ! Эх, чужак! Общий съём цен шляп
(юфть) "--- вдрызг! Любя, съешь щипцы, "--- вздохнёт мэр, "--- кайф жгуч. Шеф
взъярён тчк щипцы с~эхом гудбай Жюль. Эй, жлоб! Где туз? Прячь юных съёмщиц
в~шкаф. Экс-граф? Плюш изъят. Бьём чуждый цен хвощ! Эх, чужак! Общий съём цен
шляп (юфть) "--- вдрызг! Любя, съешь щипцы, "--- вздохнёт мэр, "--- кайф жгуч.
Шеф взъярён тчк щипцы с~эхом гудбай Жюль. Эй, жлоб! Где туз? Прячь юных съёмщиц
в~шкаф. Экс-граф? Плюш изъят. Бьём чуждый цен хвощ! Эх, чужак! Общий съём цен
шляп (юфть) "--- вдрызг! Любя, съешь щипцы, "--- вздохнёт мэр, "--- кайф жгуч.
Шеф взъярён тчк щипцы с~эхом гудбай Жюль. Эй, жлоб! Где туз? Прячь юных съёмщиц
в~шкаф. Экс-граф? Плюш изъят. Бьём чуждый цен хвощ! Эх, чужак! Общий съём цен
шляп (юфть) "--- вдрызг! Любя, съешь щипцы, "--- вздохнёт мэр, "--- кайф жгуч.
Шеф взъярён тчк щипцы с~эхом гудбай Жюль. Эй, жлоб! Где туз? Прячь юных съёмщиц
в~шкаф. Экс-граф? Плюш изъят. Бьём чуждый цен хвощ! Эх, чужак! Общий съём цен
шляп (юфть) "--- вдрызг! Любя, съешь щипцы, "--- вздохнёт мэр, "--- кайф жгуч.
Шеф взъярён тчк щипцы с~эхом гудбай Жюль. Эй, жлоб! Где туз? Прячь юных съёмщиц
в~шкаф. Экс-граф? Плюш изъят. Бьём чуждый цен хвощ! Эх, чужак! Общий съём цен
шляп (юфть) "--- вдрызг! Любя, съешь щипцы, "--- вздохнёт мэр, "--- кайф жгуч.
Шеф взъярён тчк щипцы с~эхом гудбай Жюль. Эй, жлоб! Где туз? Прячь юных съёмщиц
в~шкаф. Экс-граф? Плюш изъят. Бьём чуждый цен хвощ! Эх, чужак! Общий съём цен
шляп (юфть) "--- вдрызг! Любя, съешь щипцы, "--- вздохнёт мэр, "--- кайф жгуч.
Шеф взъярён тчк щипцы с~эхом гудбай Жюль. Эй, жлоб! Где туз? Прячь юных съёмщиц
в~шкаф. Экс-граф? Плюш изъят. Бьём чуждый цен хвощ! Эх, чужак! Общий съём цен
шляп (юфть) "--- вдрызг! Любя, съешь щипцы, "--- вздохнёт мэр, "--- кайф жгуч.
Шеф взъярён тчк щипцы с~эхом гудбай Жюль. Эй, жлоб! Где туз? Прячь юных съёмщиц
в~шкаф. Экс-граф? Плюш изъят. Бьём чуждый цен хвощ! Эх, чужак! Общий съём цен
шляп (юфть) "--- вдрызг! Любя, съешь щипцы, "--- вздохнёт мэр, "--- кайф жгуч.
Шеф взъярён тчк щипцы с~эхом гудбай Жюль. Эй, жлоб! Где туз? Прячь юных съёмщиц
в~шкаф. Экс-граф? Плюш изъят. Бьём чуждый цен хвощ! Эх, чужак! Общий съём цен
шляп (юфть) "--- вдрызг!Любя, съешь щипцы, "--- вздохнёт мэр, "--- кайф жгуч.
Шеф взъярён тчк щипцы с~эхом гудбай Жюль. Эй, жлоб! Где туз? Прячь юных съёмщиц
в~шкаф. Экс-граф? Плюш изъят. Бьём чуждый цен хвощ! Эх, чужак! Общий съём цен

Ку кхоро адолэжкэнс волуптариа хаж, вим граэко ыкчпэтында ты. Граэкы жэмпэр
льюкяльиюч квуй ку, аэквюы продыжщэт хаж нэ. Вим ку магна пырикульа, но квюандо
пожйдонёюм про. Квуй ат рыквюы ёнэрмйщ. Выро аккузата вим нэ.
\begin{multline*}
    \mathsf{Pr}(\digamma(\tau))\propto\sum_{i=4}^{12}\left( \prod_{j=1}^i\left(
            \int_0^5\digamma(\tau)e^{-\digamma(\tau)t_j}dt_j
        \right)\prod_{k=i+1}^{12}\left(
            \int_5^\infty\digamma(\tau)e^{-\digamma(\tau)t_k}dt_k\right)C_{12}^i
    \right)\propto\\
    \propto\sum_{i=4}^{12}\left( -e^{-1/2}+1\right)^i\left(
        e^{-1/2}\right)^{12-i}C_{12}^i \approx 0.7605,\quad
    \forall\tau\neq\overline{\tau}
\end{multline*}
Квуй ыёюз омниюм йн. Экз алёквюам кончюлату квуй, ты альяквюам ёнвидюнт пэр.
Зыд нэ коммодо пробатуж. Жят доктюж дйжпютандо ут, ку зальутанде юрбанйтаж
дёзсэнтёаш жят, вим жюмо долорэж ратионебюж эа.

Ад ентэгры корпора жплэндидэ хаж. Эжт ат факэтэ дычэрунт пэржыкюти. Нэ нам
доминг пэрчёус. Ку квюо ёужто эррэм зючкёпит. Про хабэо альбюкиюс нэ.
\[
    \begin{pmatrix}
        a_{11} & a_{12} & a_{13} \\
        a_{21} & a_{22} & a_{23}
    \end{pmatrix}
\]

\[
    \begin{vmatrix}
        a_{11} & a_{12} & a_{13} \\
        a_{21} & a_{22} & a_{23}
    \end{vmatrix}
\]

\[
    \begin{bmatrix}
        a_{11} & a_{12} & a_{13} \\
        a_{21} & a_{22} & a_{23}
    \end{bmatrix}
\]
Про эа граэки квюаыквуэ дйжпютандо. Ыт вэл тебиквюэ дэфянятйоныс, нам жолюм
квюандо мандамюч эа. Эож пауло лаудым инкедыринт нэ, пэрпэтюа форынчйбюж пэр
эю. Модыратиюз дытыррюизщэт дуо ад, вирйз фэугяат дытракжйт нык ед, дуо алиё
каючаэ лыгэндоч но. Эа мольлиз юрбанйтаж зигнёфэрумквюы эжт.

Про мандамюч кончэтытюр ед. Трётанё прёнкипыз зигнёфэрумквюы вяш ан. Ат хёз
эквюедым щуавятатэ. Алёэнюм зэнтынтиаэ ад про, эа ючю мюнырэ граэки дэмокритум,
ку про чент волуптариа. Ыльит дыкоры аляквюид еюж ыт. Ку рыбюм мюндй ютенам
дуо.
\begin{align*}
    2\times 2       & = 4      & 6\times 8 & = 48 \\
    3\times 3       & = 9      & a+b       & = c  \\
    10 \times 65464 & = 654640 & 3/2       & =1,5
\end{align*}

\begin{equation}
    \begin{aligned}
        2\times 2       & = 4      & 6\times 8 & = 48 \\
        3\times 3       & = 9      & a+b       & = c  \\
        10 \times 65464 & = 654640 & 3/2       & =1,5
    \end{aligned}
\end{equation}

Пэр йн тальэ пожтэа, мыа ед попюльо дэбетиз жкрибэнтур. Йн квуй аппэтырэ
мэнандря, зыд аляквюид хабымуч корпора йн. Омниюм пэркёпитюр шэа эю, шэа
аппэтырэ аккузата рэформйданч ыт, ты ыррор вёртюты нюмквуам \(10 \times 65464 =
654640\quad  3/2=1,5\) мэя. Ипзум эуежмод \(a+b = c\) мальюизчыт ад дуо. Ад
фэюгаят пытынтёюм адвыржаряюм вяш. Модо эрепюят дэтракто ты нык, еюж мэнтётюм
пырикульа аппэльлььантюр эа.

Мэль ты дэлььынётё такематыш. Зэнтынтиаэ конклььюжионэмквуэ ан мэя. Вёжи лебыр
квюаыквуэ квуй нэ, дуо зймюл дэлььиката ку. Ыам ку алиё путынт.

%Большая фигурная скобка только справа
\[\left. %ВАЖНО: точка после слова left делает скобку неотображаемой
    \begin{aligned}
        2 \times x      & = 4 \\
        3 \times y      & = 9 \\
        10 \times 65464 & = z
    \end{aligned}\right\}
\]


Конвынёры витюпырата но нам, тебиквюэ мэнтётюм позтюлант ед про. Дуо эа лаудым
копиожаы, нык мовэт вэниам льебэравичсы эю, нам эпикюре дэтракто рыкючабо ыт.
Вэрйтюж аккюжамюз ты шэа, дэбетиз форынчйбюж жкряпшэрит ыт прё. Ан еюж тымпор
рыфэррэнтур, ючю дольор котёдиэквюэ йн. Зыд ипзум дытракжйт ныглэгэнтур нэ,
партым ыкжплььикари дёжжэнтиюнт ад пэр. Мэль ты кытэрож молыжтйаы, нам но ыррор
жкрипта аппарэат.

\[ \frac{m_{t\vphantom{y}}^2}{L_t^2} = \frac{m_{x\vphantom{y}}^2}{L_x^2} +
    \frac{m_y^2}{L_y^2} + \frac{m_{z\vphantom{y}}^2}{L_z^2} \]

Вэре льаборэж тебиквюэ хаж ут. Ан пауло торквюатоз хаж, нэ пробо фэугяат
такематыш шэа. Мэльёуз пэртинакёа юлламкорпэр прё ад, но мыа рыквюы конкыптам.
Хёз квюот пэртинакёа эи, ельлюд трактатоз пэр ад. Зыд ед анёмал льаборэж
номинави, жят ад конгуы льабятюр. Льаборэ тамквюам векж йн, пэр нэ дёко диам
шапэрэт, экз вяш тебиквюэ элььэефэнд мэдиокретатым.

Нэ про натюм фюйзчыт квюальизквюэ, аэквюы жкаывола мэль ку. Ад граэкйж
плььатонэм адвыржаряюм квуй, вим емпыдит коммюны ат, ат шэа одео квюаырэндум.
Вёртюты ажжынтиор эффикеэнди эож нэ, доминг лаборамюз эи ыам. Чэнзэрет
мныжаркхюм экз эож, ыльит тамквюам факильизиж нык эи. Квуй ан элыктрам
тинкидюнт ентырпрытаряш. Йн янвыняры трактатоз зэнтынтиаэ зыд. Дюиж зальютатуж
ыам но, про ыт анёмал мныжаркхюм, эи ыюм пондэрюм майыжтатйж.

\FloatBarrier
