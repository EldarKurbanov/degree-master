\thispagestyle{empty}
\begin{center}
\thesisOrganization
\end{center}

\vspace{0.5cm}

\hfill\begin{minipage}{0.4\textwidth}
        УТВЕРЖДАЮ: \\
	Зав.~каф.~КНЭМ \\
	\underline{\hspace{3cm}} В.А. Клячин\\
	<<\underline{\hspace{1cm}}>>\underline{\hspace{3cm}} 2022г.
\end{minipage}

\vspace{1cm}

\begin{center}
ЗАДАНИЕ

На выполнение выпускной квалификационной работы

(магистерской работы)

Студента Курбанова Эльдара Ровшановича группы МОСм-201
\end{center}

\vspace{0.3cm}

\noindent Тема: Система управления и формирования поведенческой стратегии автономного мобильного
робота на основе визуального анализа окружающего пространства.

\noindent Цель: Создание и тестирование системы управления роботом на основе визуального анализа окружающего пространства, подбор комплектующих робота, а также сбор тренировочных данных для управляющей нейронной сети на основе обучения с подкреплением.

\noindent Основные задачи:

\begin{enumerate}
	\item Исследовать предметную область робототехники;
	\item Изучить существующие известные аналоги;
	\item На основе выданных требований разработать схему управления роботом и соответствующее ПО;
	\item Протестировать созданное изделие;
	\item Собрать тренировочные данные для нейронной сети.
\end{enumerate}	

\noindent Основные этапы:

\begin{enumerate}
	\item Изучение предметной области робототехники;
	\item Изучение поставленных к роботу требований;
	\item Подбор среды выполнения ПО;
	\item Подбор комплектующих для сбора необходимых данных и движения;
	\item Рассмотрение аналогов;
	\item Изучение фреймворка Robot Operating System;
	\item Создание схемы компоновки и электрификации оборудования;
	\item Создание робота, на котором будет проводиться разработка;
	\item Установка и настройка ПО;
	\item Реализация Robot Navigation Stack;
	\item Создание системы для отладки и тестирования робота;
	\item Тестирование созданного изделия;
	\item Сбор тренировочных данных для нейронной сети.
\end{enumerate}

\vspace{0.3cm}

\noindent Рекомендуемая литература:

\begin{enumerate}
	\item Joseph, L. Mastering ROS for Robotics Programming: Design, build, and simulate complex robots using the Robot Operating System, 2nd Edition [Текст] / L. Joseph, J. Cacace. — Packt Publishing, 2018. — 580 с. — Текст: непосредственный;
	
	\item Электронный ресурс: официальная документация ROS : сайт / Willow Garage, Inc. — URL: \url{http://wiki.ros.org/Documentation}. - Текст : электронный;
	
	\item Real-Time Loop Closure in 2D LIDAR SLAM / W. Hess [и др.] // 2016 IEEE International Conference on Robotics and Automation (ICRA). — Текст: электронный. — 2016. — С. 1271—1278. — URL: \url{https://storage.googleapis.com/pub-tools-public-publication-data/pdf/45466.pdf} (дата обр. 11.06.2022).
\end{enumerate}

\newcommand\tline[2]{$\underset{\text{#1}}{\text{\underline{\hspace{#2}}}}$}
\vspace{2cm}

Дата выдачи \underline{\hspace{4cm}} Срок исполнения \underline{\hspace{4.5cm}}

Руководитель: \tline{(подпись)}{5cm} \hspace{0.3cm} \tline{(ФИО)}{6.5cm}

\vspace{1cm}

Задание принял к исполнению \tline{(подпись)}{5cm}