
{\actuality} Актуальность данной работы обусловлена всеобщей автоматизацией и <<роботизацией>> деятельности человека в условиях современной реальности~\cite{robotization}. Решение поставленной задачи позволит создать робота, умеющего исследовать разного вида помещения, распознавая опасные объекты, попавшие в его поле зрения. Это позволит не только делегировать на робота задачи по обнаружению, но и снизит риски травмирования или гибели человека в случае, когда объект представляет какую-либо угрозу здоровью.


{\progress} В настоящий момент поставленная данной работой задача формирования поведенческой стратегии и управления роботом решена не полностью. Требуются значительные улучшения на стороне нейронной сети, управляющей роботом, однако её создание и улучшение не являлись целью данной работы.


{\aim} данной работы является подготовка аппаратной части робота для корректного движения, а также сбор необходимых данных для обучения нейронной сети, которая будет управлять данным роботом.


Для~достижения поставленной цели необходимо было решить следующие {\tasks}:
\begin{enumerate}[beginpenalty=10000] % https://tex.stackexchange.com/a/476052/104425
  \item Исследовать предметную область робототехники\footnote{Робототехника не изучалась на протяжении всего курса обучения в университете.} (аппаратную и программную часть);
  \item Изучить существующие известные аналоги (в~т.\:ч зарубежные);
  \item Подобрать совместимый и недорогой программно-аппаратный комплекс, подходящий под задачи запуска нейронной сети;
  \item Протестировать работоспособность и совместимость аппаратной и программной частей.
\end{enumerate}


{\novelty}
\begin{enumerate}[beginpenalty=10000] % https://tex.stackexchange.com/a/476052/104425
  \item Впервые в России создана роботизированная платформа для работы с нейронной сетью на основе обучения с подкреплением на базе ROS и Jetson Xavier NX\footnote{Автором работы не было найдено подобной разработки в открытых источниках сети Интернет}.
\end{enumerate}

{\influence} данной работы заключается в создании мобильной программно-аппаратной платформы для автономной работы нейронной сети, реализующей поведенческую стратегию робота в заранее неизвестном замкнутом пространстве.

{\methods} При разработке данного робота использовались такие методы эмпирического исследования, как наблюдение и эксперимент, а к методам теоретического исследования - анализ, синтез и восхождение от абстрактного к конкретному.

{\defpositions}
\begin{enumerate}[beginpenalty=10000] % https://tex.stackexchange.com/a/476052/104425
  \item Если робот был запущен без ошибок, то он будет предоставлять все необходимые данные управляющей нейронной сети;
  \item Если роботу были предоставлены управляющие сигналы для движения, то он с минимальной задержкой и должен исполнить их.
\end{enumerate}

{\reliability} полученного результата работы обеспечивается соответствующими экспериментами. \ Результаты находятся в соответствии с договорённостями научного консультанта и автора данной работы.


{\probation}
Основные результаты работы докладывались~на:
Третьей международной конференции молодых учёных по обработке информации (YSIP3, 2019), глобальной технологической конференции 2020 года (GTC 2020 by NVIDIA).

{\contribution} Автор принимал активное участие не только в разработке программно-аппаратного комплекса робота, но и помогал в сборе обучающих данных для нейронной сети на базе обучения с подкреплением, которая будет управлять данным роботом.

\ifnumequal{\value{bibliosel}}{0}
{%%% Встроенная реализация с загрузкой файла через движок bibtex8. (При желании, внутри можно использовать обычные ссылки, наподобие `\cite{vakbib1,vakbib2}`).
    {\publications} Основные результаты по теме диссертации изложены
    в~XX~печатных изданиях,
    X из которых изданы в журналах, рекомендованных ВАК,
    X "--- в тезисах докладов.
}%
{%%% Реализация пакетом biblatex через движок biber
    \begin{refsection}[bl-author, bl-registered]
        % Это refsection=1.
        % Процитированные здесь работы:
        %  * подсчитываются, для автоматического составления фразы "Основные результаты ..."
        %  * попадают в авторскую библиографию, при usefootcite==0 и стиле `\insertbiblioauthor` или `\insertbiblioauthorgrouped`
        %  * нумеруются там в зависимости от порядка команд `\printbibliography` в этом разделе.
        %  * при использовании `\insertbiblioauthorgrouped`, порядок команд `\printbibliography` в нём должен быть тем же (см. biblio/biblatex.tex)
        %
        % Невидимый библиографический список для подсчёта количества публикаций:
        \printbibliography[heading=nobibheading, section=1, env=countauthorvak,          keyword=biblioauthorvak]%
        \printbibliography[heading=nobibheading, section=1, env=countauthorwos,          keyword=biblioauthorwos]%
        \printbibliography[heading=nobibheading, section=1, env=countauthorscopus,       keyword=biblioauthorscopus]%
        \printbibliography[heading=nobibheading, section=1, env=countauthorconf,         keyword=biblioauthorconf]%
        \printbibliography[heading=nobibheading, section=1, env=countauthorother,        keyword=biblioauthorother]%
        \printbibliography[heading=nobibheading, section=1, env=countregistered,         keyword=biblioregistered]%
        \printbibliography[heading=nobibheading, section=1, env=countauthorpatent,       keyword=biblioauthorpatent]%
        \printbibliography[heading=nobibheading, section=1, env=countauthorprogram,      keyword=biblioauthorprogram]%
        \printbibliography[heading=nobibheading, section=1, env=countauthor,             keyword=biblioauthor]%
        \printbibliography[heading=nobibheading, section=1, env=countauthorvakscopuswos, filter=vakscopuswos]%
        \printbibliography[heading=nobibheading, section=1, env=countauthorscopuswos,    filter=scopuswos]%
        %
        \nocite{*}%
        %
        {\publications} Основные результаты по теме работы изложены в~\arabic{citeauthor}~печатных изданиях,
        %\arabic{citeauthorvak} из которых изданы в журналах, рекомендованных ВАК\sloppy%
        \ifnum \value{citeauthorscopuswos}>0%
            , \arabic{citeauthorscopuswos} "--- в~периодических научных журналах, индексируемых Web of~Science и Scopus\sloppy%
        \fi%
        \ifnum \value{citeauthorconf}>0%
            , \arabic{citeauthorconf} "--- в~тезисах докладов.
        \else%
            .
        \fi%
        %\ifnum \value{citeregistered}=1%
        %    \ifnum \value{citeauthorpatent}=1%
        %        Зарегистрирован \arabic{citeauthorpatent} патент.
        %    \fi%
        %    \ifnum \value{citeauthorprogram}=1%
        %        Зарегистрирована \arabic{citeauthorprogram} программа для ЭВМ.
        %    \fi%
        %\fi%
        %\ifnum \value{citeregistered}>1%
        %    Зарегистрированы\ %
        %    \ifnum \value{citeauthorpatent}>0%
        %    \formbytotal{citeauthorpatent}{патент}{}{а}{}\sloppy%
        %    \ifnum \value{citeauthorprogram}=0 . \else \ и~\fi%
        %    \fi%
        %    \ifnum \value{citeauthorprogram}>0%
        %    \formbytotal{citeauthorprogram}{программ}{а}{ы}{} для ЭВМ.
        %    \fi%
        %\fi%
        % К публикациям, в которых излагаются основные научные результаты диссертации на соискание учёной
        % степени, в рецензируемых изданиях приравниваются патенты на изобретения, патенты (свидетельства) на
        % полезную модель, патенты на промышленный образец, патенты на селекционные достижения, свидетельства
        % на программу для электронных вычислительных машин, базу данных, топологию интегральных микросхем,
        % зарегистрированные в установленном порядке.(в ред. Постановления Правительства РФ от 21.04.2016 N 335)
    \end{refsection}%
    \begin{refsection}[bl-author, bl-registered]
        % Это refsection=2.
        % Процитированные здесь работы:
        %  * попадают в авторскую библиографию, при usefootcite==0 и стиле `\insertbiblioauthorimportant`.
        %  * ни на что не влияют в противном случае
        \nocite{vakbib2}%vak
        \nocite{patbib1}%patent
        \nocite{progbib1}%program
        \nocite{bib1}%other
        \nocite{confbib1}%conf
    \end{refsection}%
        %
        % Всё, что вне этих двух refsection, это refsection=0,
        %  * для диссертации - это нормальные ссылки, попадающие в обычную библиографию
        %  * для автореферата:
        %     * при usefootcite==0, ссылка корректно сработает только для источника из `external.bib`. Для своих работ --- напечатает "[0]" (и даже Warning не вылезет).
        %     * при usefootcite==1, ссылка сработает нормально. В авторской библиографии будут только процитированные в refsection=0 работы.
}

%При использовании пакета \verb!biblatex! будут подсчитаны все работы, добавленные
%в файл \verb!biblio/author.bib!. Для правильного подсчёта работ в~различных
%системах цитирования требуется использовать поля:
%\begin{itemize}
%        \item \texttt{authorvak} если публикация индексирована ВАК,
%        \item \texttt{authorscopus} если публикация индексирована Scopus,
%        \item \texttt{authorwos} если публикация индексирована Web of Science,
%        \item \texttt{authorconf} для докладов конференций,
%        \item \texttt{authorpatent} для патентов,
%        \item \texttt{authorprogram} для зарегистрированных программ для ЭВМ,
%        \item \texttt{authorother} для других публикаций.
%\end{itemize}
%Для подсчёта используются счётчики:
%\begin{itemize}
%        \item \texttt{citeauthorvak} для работ, индексируемых ВАК,
%        \item \texttt{citeauthorscopus} для работ, индексируемых Scopus,
%        \item \texttt{citeauthorwos} для работ, индексируемых Web of Science,
%        \item \texttt{citeauthorvakscopuswos} для работ, индексируемых одной из трёх баз,
%        \item \texttt{citeauthorscopuswos} для работ, индексируемых Scopus или Web of~Science,
%        \item \texttt{citeauthorconf} для докладов на конференциях,
%        \item \texttt{citeauthorother} для остальных работ,
%        \item \texttt{citeauthorpatent} для патентов,
%        \item \texttt{citeauthorprogram} для зарегистрированных программ для ЭВМ,
%        \item \texttt{citeauthor} для суммарного количества работ.
%\end{itemize}
% Счётчик \texttt{citeexternal} используется для подсчёта процитированных публикаций;
% \texttt{citeregistered} "--- для подсчёта суммарного количества патентов и программ для ЭВМ.
