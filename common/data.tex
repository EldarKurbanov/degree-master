%%% Основные сведения %%%
\newcommand{\thesisAuthorLastName}{Курбанов}
\newcommand{\thesisAuthorOtherNames}{Эльдар Ровшанович}
\newcommand{\thesisAuthorInitials}{Э.\,Р.}
\newcommand{\thesisAuthor}             % Диссертация, ФИО автора
{%
    \texorpdfstring{% \texorpdfstring takes two arguments and uses the first for (La)TeX and the second for pdf
        \thesisAuthorLastName~\thesisAuthorOtherNames% так будет отображаться на титульном листе или в тексте, где будет использоваться переменная
    }{%
        \thesisAuthorLastName, \thesisAuthorOtherNames% эта запись для свойств pdf-файла. В таком виде, если pdf будет обработан программами для сбора библиографических сведений, будет правильно представлена фамилия.
    }
}
\newcommand{\thesisAuthorShort}        % Диссертация, ФИО автора инициалами
{\thesisAuthorInitials~\thesisAuthorLastName}
%\newcommand{\thesisUdk}                % Диссертация, УДК
%{\fixme{xxx.xxx}}
\newcommand{\thesisTitle}              % Диссертация, название
{СИСТЕМА УПРАВЛЕНИЯ И ФОРМИРОВАНИЯ ПОВЕДЕНЧЕСКОЙ СТРАТЕГИИ АВТОНОМНОГО МОБИЛЬНОГО РОБОТА НА ОСНОВЕ ВИЗУАЛЬНОГО АНАЛИЗА ОКРУЖАЮЩЕГО ПРОСТРАНСТВА}
\newcommand{\thesisSpecialtyNumber}    % Диссертация, специальность, номер
{02.04.03}
\newcommand{\thesisSpecialtyTitle}     % Диссертация, специальность, название (название взято с сайта ВАК для примера)
{Математическое обеспечение и администрирование информационных систем}
%% \newcommand{\thesisSpecialtyTwoNumber} % Диссертация, вторая специальность, номер
%% {\fixme{XX.XX.XX}}
%% \newcommand{\thesisSpecialtyTwoTitle}  % Диссертация, вторая специальность, название
%% {\fixme{Теория и~методика физического воспитания, спортивной тренировки,
%% оздоровительной и~адаптивной физической культуры}}
\newcommand{\thesisDegree}             % Диссертация, ученая степень
{магистра}
\newcommand{\thesisDegreeShort}        % Диссертация, ученая степень, краткая запись
{маг.}
\newcommand{\thesisCity}               % Диссертация, город написания диссертации
{Волгоград}
\newcommand{\thesisYear}               % Диссертация, год написания диссертации
{\the\year}
\newcommand{\thesisOrganization}       % Диссертация, организация
{Министерство науки и высшего образования Российской Федерации Федеральное государственное автономное образовательное учреждение высшего образования <<Волгоградский государственный университет>> Кафедра~\textit{Компьютерных наук и экспериментальной математики}}
\newcommand{\thesisOrganizationShort}  % Диссертация, краткое название организации для доклада
{Система управления роботом}

\newcommand{\thesisInOrganization}     % Диссертация, организация в предложном падеже: Работа выполнена в ...
{Министерстве науки и высшего образования Российской Федерации Федеральное государственном автономном образовательном учреждении высшего образования <<Волгоградский государственный университет>> Кафедры~\textit{Компьютерных наук и экспериментальной математики}}

%% \newcommand{\supervisorDead}{}           % Рисовать рамку вокруг фамилии
\newcommand{\supervisorFio}              % Научный руководитель, ФИО
{Клячин Владимир Александрович}
\newcommand{\supervisorRegalia}          % Научный руководитель, регалии
{доктор физико-математических наук,~профессор}
\newcommand{\supervisorJobPost}          % Научный руководитель, должность
{заведующий кафедрой КНЭМ}
\newcommand{\supervisorFioShort}         % Научный руководитель, ФИО
{В.\,В.~Клячин}
\newcommand{\supervisorRegaliaShort}     % Научный руководитель, регалии
{д.~ф.~-м.~н.,~проф.}
\newcommand{\supervisorJobPostShort}   % Научный руководитель, должность
{зав.~кафедрой КНЭМ}

%% \newcommand{\supervisorTwoDead}{}        % Рисовать рамку вокруг фамилии
\newcommand{\supervisorTwoFio}           % Второй научный руководитель, ФИО
{Гордеев Алексей Юрьевич}
%% \newcommand{\supervisorTwoRegalia}       % Второй научный руководитель, регалии
%% {\fixme{доктор физико-математических наук, профессор}}
\newcommand{\supervisorTwoJobPost}          % Научный руководитель, должность
{старший преподаватель кафедры КНЭМ}
\newcommand{\supervisorTwoFioShort}      % Второй научный руководитель, ФИО
{А.\,Ю.~Гордеев}
%% \newcommand{\supervisorTwoRegaliaShort}  % Второй научный руководитель, регалии
%% {\fixme{д.~ф.~-м.~н.,~проф.}}
\newcommand{\supervisorTwoJobPostShort}          % Научный руководитель, должность
{ст.~преп.,~каф.~КНЭМ}

%% \newcommand{\opponentOneFio}           % Оппонент 1, ФИО
%% {\fixme{Фамилия Имя Отчество}}
%% \newcommand{\opponentOneRegalia}       % Оппонент 1, регалии
%% {\fixme{доктор физико-математических наук, профессор}}
%% \newcommand{\opponentOneJobPlace}      % Оппонент 1, место работы
%% {\fixme{Не очень длинное название для места работы}}
%% \newcommand{\opponentOneJobPost}       % Оппонент 1, должность
%% {\fixme{старший научный сотрудник}}

%% \newcommand{\opponentTwoFio}           % Оппонент 2, ФИО
%% {\fixme{Фамилия Имя Отчество}}
%% \newcommand{\opponentTwoRegalia}       % Оппонент 2, регалии
%% {\fixme{кандидат физико-математических наук}}
%% \newcommand{\opponentTwoJobPlace}      % Оппонент 2, место работы
%% {\fixme{Основное место работы c длинным длинным длинным длинным названием}}
%% \newcommand{\opponentTwoJobPost}       % Оппонент 2, должность
%% {\fixme{старший научный сотрудник}}

%% \newcommand{\opponentThreeFio}         % Оппонент 3, ФИО
%% {\fixme{Фамилия Имя Отчество}}
%% \newcommand{\opponentThreeRegalia}     % Оппонент 3, регалии
%% {\fixme{кандидат физико-математических наук}}
%% \newcommand{\opponentThreeJobPlace}    % Оппонент 3, место работы
%% {\fixme{Основное место работы c длинным длинным длинным длинным названием}}
%% \newcommand{\opponentThreeJobPost}     % Оппонент 3, должность
%% {\fixme{старший научный сотрудник}}

%% \newcommand{\leadingOrganizationTitle} % Ведущая организация, дополнительные строки. Удалить, чтобы не отображать в автореферате
%% {\fixme{Федеральное государственное бюджетное образовательное учреждение высшего
%% профессионального образования с~длинным длинным длинным длинным названием}}

\newcommand{\defenseDate}              % Защита, дата
{16 июня 2022~г.~в~12 часов}
%% \newcommand{\defenseCouncilNumber}     % Защита, номер диссертационного совета
%% {\fixme{Д\,123.456.78}}
\newcommand{\defenseCouncilTitle}      % Защита, учреждение диссертационного совета
{Федеральное государственное автономное образовательное учреждение высшего образования <<Волгоградский государственный университет>>}
\newcommand{\defenseCouncilAddress}    % Защита, адрес учреждение диссертационного совета
{400062, Волгоградская область, г. Волгоград, просп. Университетский, д.100}
\newcommand{\defenseCouncilPhone}      % Телефон для справок
{+7~(8442)~46-02-63}

%% \newcommand{\defenseSecretaryFio}      % Секретарь диссертационного совета, ФИО
%% {\fixme{Фамилия Имя Отчество}}
%% \newcommand{\defenseSecretaryRegalia}  % Секретарь диссертационного совета, регалии
%% {\fixme{д-р~физ.-мат. наук}}            % Для сокращений есть ГОСТы, например: ГОСТ Р 7.0.12-2011 + http://base.garant.ru/179724/#block_30000

%% \newcommand{\synopsisLibrary}          % Автореферат, название библиотеки
%% {\fixme{Название библиотеки}}
%% \newcommand{\synopsisDate}             % Автореферат, дата рассылки
%% {\fixme{DD mmmmmmmm}\the\year~года}

% To avoid conflict with beamer class use \providecommand
\providecommand{\keywords}%            % Ключевые слова для метаданных PDF диссертации и автореферата
{}
